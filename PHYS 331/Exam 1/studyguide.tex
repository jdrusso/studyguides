\documentclass[a4paper, 11pt]{article}
\usepackage{comment} % enables the use of multi-line comments (\ifx \fi)
\usepackage{fullpage} % changes the margin
\usepackage{esvect}
\usepackage{color,soul}

\begin{document}
%Header-Make sure you update this information!!!!
\noindent
\setlength\parindent{0pt}

\large\textbf{Study Guide} \hfill \textbf{JD Russo} \\
\normalsize PHYS 311  \hfill Exam Date: 09/14/2016 \\%\hfill Teammates: Student1, Student2 \\
Prof. Amthor

%TODO: have this add to an appendix
\newcommand{\notecard}[1]{\textbf{Notecard:} #1\\}

\section*{Chapters 1-4}
% \begin{itemize}
  \subsection{Vector operations (dot, cross, gradient, magnitude), including
  familiarity with polar coordinates.}

  These should be fairly straightforward.


  \notecard{Basic vector equations for the above operations.}


  \subsection{Newtons Second Law for the determination of equations of motion}

  Changes with coordinate system

  \notecard{$ \vec{F}_{net} = m \vec{a} $}
  \notecard{$ \vec{F}_{r} = m (\ddot{r} - r \dot{\phi}^2) $}
  \notecard{$ \vec{F}_{\phi} = m ( r \ddot{\phi} + 2 \dot{r} \dot{\phi}) $}


  \subsection{Inertial and noninertial reference frames}

  An inertial reference frame is nonaccelerating, which also implies nonrotating.

  \subsection{Simple Harmonic Oscillator}

  \hl{What about it?}

  \subsection{Solutions of our favorite differential equations}

  $\ddot{\phi} = -\omega^2 \phi$\\
  $\phi = A \sin{\omega t} + B \cos{\omega t}$ \\
  $ = \tilde{A}e^{i \omega \left(t + \phi_0 \right)}$



  \subsection{Separation of variables technique}
  Straightforward


  \subsection{Drag forces}

  \textbf{Linear}\\
  % $V_y(t) = A e^{-b/m - t} + \frac{mg}{b}$
  % $\dot{v_x} = - k v_x$
  $F_{drag} = m \dot{v} = - b v$\\
  $v(t) = A e^{-kt}$

\quad

  \textbf{Quadratic}\\
  $F_{drag} = m \dot{v} = c v^2$\\
  Separation of variables on the above leads you to $v = \frac{v_0}{1 + c v_0 t / m}$

\quad

  \notecard{$F_{drag} = m \dot{v} = - b v$}
  \notecard{$v(t) = A e^{-kt}$}
  \notecard{$F_{drag} = m \dot{v} = c v^2$}
  \notecard{$v = \frac{v_0}{1 + c v_0 t / m}$}


  \subsection{Investigation of the qualities of motion in the limit, whether or not
  analytical solutions are possible}

  \hl{What?}


  \subsection{Use of approximations, i.e. Taylor expansion for oscillations about an
  equilibrium position}

  \hl{Oscillation around equilibrium? What is this expansion?}

  Casting things into a binomial expansion is often helpful! Think of what Ned
  said about having a main term and a correcting term.

  \notecard{Taylor Series - $f(z) = f(a) + f'(a)(z-a) + \frac{1}{2!}f''(a)(z-a)^2$}
  \notecard{Binomial expansion - $(1+z)^n = 1 + nz + \frac{n(n-1)}{2!}z^2 + ...$}


  \subsection{Linear and Angular Momentum - Definitions, conservations, and rate of change
  given external forces/torques}

  \textbf{Momenta}\\
  Linear: $\vec{p} = m \vec{v}$\\
  % Angular Momentum: $\vec{I} = m \vec{\omega}$
  Angular: $\vec{\ell} = r \times p$

  Remember that force is the derivative of momentum, so the derivative of angular
  momentum has the force acting on the particle in it (product rule works for
  cross product).

  \quad

  \textbf{Conservation}\\
  Linear: If no external \textbf{forces} are acting, $\vec{p}$ stays constant.\\
  Angular: If no external \textbf{torques} are present, the total angular
  momentum $ L = \sum r_\alpha \times p_\alpha$ is constant. \\
  These are both evident in the below rates of change.

  \quad

  \textbf{Rates of Change}\\
  Linear: $\dot{p} = F$ \\
  Angular: $\vec{\dot{\ell}} = r \times F = \Gamma$  (where $\Gamma = $ torque)


  \subsection{Moment of inertia and CoM for both point mass (summation form) and
  continuous form (integral form) distributions}

  Center of mass position $R = \frac{1}{M} \sum_{i=1}{N} m_i r_i$\\
  $R = \frac{1}{M} \int r dm = \frac{1}{M} \int \rho r dV$ where $\rho$ is the
  density of the object

  \quad

  \textbf{Moments of inertia}
  \begin{itemize}
    \item Component of angular momentum $L_x = I \omega$ where $\omega$ is the
    angular rotation about the $x$ axis
    \item Uniform disk rotating about its axis: $I = 1/2 M R^2$
    \item Uniform solid sphere: $I = 2/5 M R^2$
    \item Generally: $I = \sum m_\alpha p_\alpha^2$ where $p_\alpha$ is the distance
      of the mass $m_\alpha$ from the axis of rotation.

  \end{itemize}

  \notecard{$I = 1/2 M R^2$}
  \notecard{$I = 2/5 M R^2$}
  \notecard{$I = \sum m_\alpha p_\alpha^2$}

  \hl{Angular momentum about CoM?}


  \subsection{Work and Kinetic Energy - Work theorem}

  \textbf{Work-KE Theorem: } $\delta KE \equiv KE_2 - KE_1 = \sum F \cdot dr$\\
  $ W = \int F \cdot d\vec{r}$


  \subsection{Conservative and non-conservative forces (define, identify, use)}

  \textbf{Define: } A conservative force does the same work for all paths,
  and has F dependent only on R.

  \textbf{Identify: } Check if $\nabla \times F = 0$

  \textbf{Use: } For a conservative force, you can pick an easy path to do calculations.


  \subsection{Derivation of potential function, and relation to the force}

  Force is the gradient of potential energy. \\
  Review derivation (p. 116)


  \subsection{Energy of multiple particles given a potential}

  \hl{What?}

  \subsection{Spherical polar coordinates}
  \hl{Check formulas in back of book}



  \section*{Chapter 5}

  \subsection{Harmonic Oscillator solutions in sinusoidal and exponential forms,
  finding equations of motion based on initial conditions.}

  \textbf{Harmonic oscillator solutions}\\
  $F_x = -k x$\\
  $U = \frac{1}{2}kx^2$  -- This holds true for any sufficiently small displacement
  from a stable equilibrium. You can prove this by taking a Taylor expansion
  of a generic U(x), where you'll see it comes out to this. (p. 162)\\

  $\ddot{x} = - \frac{k}{m}x = -\omega^2 x$\\
  $\omega^2 = \frac{k}{m}$\\

  The solutions to this are of the form $x(t) = C_1e^{i \omega t} + C_2e^{- i \omega t}$.

  Using Euler's Identity (see next section), we can turn this into
  $B_1 \cos{\omega t} + B_2 \sin{\omega t}$, where $B_1$ = $C_1 + C_2$ and
  $B_2 = i(C_1 - C_2) $

  \hl{Finding equations of motion from initial conditions?}


  \subsection{Complex exponentials and Euler relations}
  \hl{Review homework problems on this}\\
  \notecard{$e^{\pm i \omega t} = \cos{\omega t} \pm i \sin{\omega t}$}
  \notecard{$\cos{\theta} = \frac{1}{2}\left( e^{i \theta} + e^{-i \theta} \right)$}
  \notecard{$\sin{\theta} = \frac{1}{2i}\left( e^{i \theta} - e^{-i \theta} \right)$}


  \subsection{Damped Oscillators}
  Adding a Hooke's law force $-kx$ and a resistive force $-b\dot{x}$ yields an
  equation for a damped 1D oscillator: $m\ddot{x} + b\dot{x} + kx = 0$.

  We define constants $\frac{b}{m} = 2 \beta$ and $\omega_0 = \sqrt{\frac{k}{m}}$
  to get an equation of motion $r^2 + 2 \beta r + \omega_0^2 = 0.$  \\
  This has solutions $r = -\beta \pm \sqrt{\beta^2 - \omega_0^2}$ \\

  \textbf{Underdamped: } A system is underdamped if $\beta < \omega_0$, meaning
  the amplitude of its oscillations will slowly decrease.

  \textbf{Overdamped: } A system is overdamped if $\beta > \omega_0$, meaning it
  will move out to a maximum displacement once and them asymptotically approach 0 without
  dipping back below it.

  \textbf{Critically Damped: } A system is critically damped if $\beta = \omega_0$.
  In this case, \textbf{it will most quickly settle to its equilibrium state}.

  Decay parameter, i.e. how quickly the motion dies out, increases linearly in the
  underdamped regime, reaches a maximum at critical damping, then decreases
  exponentially in the undamped regime as the damping constant $\beta$ goes to infinity.

  \notecard{$\frac{b}{m} = 2 \beta$, $\omega_0 = \sqrt{\frac{k}{m}}$}
  \notecard{$r^2 + 2 \beta r + \omega_0^2 = 0.$}
  \notecard{$r_1 = -\beta \pm \sqrt{\beta^2 - \omega_0^2}$}


  \subsection{Driven Damped Harmonic Oscillator}
  \hl{Review bottle in a bucket}

  By defining the driving force $F(t)$ as a force per unit mass $f(t)$, we can
  write the equation for a D.D.H.O. as $\ddot{x} + 2 \beta \dot{x} + \omega_0^2 = f(t)$

  \notecard{$\ddot{x} + 2 \beta \dot{x} + \omega_0^2 = f(t)$}


  \subsubsection{General solution of D.E.}

  The \textbf{particular solution} of this system is the solution to $x_p(t) = f$.
  This is in contrast to the \textbf{homogeneous solution} $x_h(t)$, which has
  the form $x_h(t) = C_1 e^{r_1 t} + C_2 e^{r_2 t}$. You can see both exponentials
  here die out as $t \rightarrow \infty$, so this is the transient response.\\

  These solutions obey superposition.\\

  The general solution (p. 184) is
  $x(t) = A\cos(\omega t - \delta) + C_1 e^{r_1 t} + C_2 e^{r_2 t}$. This has
  two clear parts, the transient exponential and the steady-state sinusoidal
  terms. The long-term behavior is dominated by the cosine term.\\

  \textbf{This only applies to systems where the restoring force and resistive
  force are linear.}\\

  \notecard{$x(t) = A\cos(\omega t - \delta) + C_1 e^{r_1 t} + C_2 e^{r_2 t}$}


  \subsubsection{Resonance}

  The long-term behavior of the oscillator after the transience dies out is
  $x(t) = A\cos(\omega t - \delta)$. The amplitude of this is given by
  $A^2 = \frac{f_0^2}{(\omega_0^2 - \omega^2)^2 + 4\beta^2\omega^2}$.\\

  From this, we can see if the difference between the driving frequency $\omega$
  and the natural frequency $\omega_0$ is large,
  then the amplitude of the oscillations will be very small. However, as
  the $\omega$ approaches $\omega_0$, we approach resonance, where the amplitude
  grows to a maximum.

  \notecard{$A^2 = \frac{f_0^2}{(\omega_0^2 - \omega^2)^2 + 4\beta^2\omega^2}$}


  \section*{Chapter 12}

  \subsection{Driven damped pendulum}

  This describes a damped pendulum driven by a sinusoidal force $F(t) = F_0 \cos{\omega t}$.\\
  $\ddot{\phi} + 2 \beta \dot{\phi} + \omega_0^2 \sin{\phi} = \gamma \omega_0^2 \cos{\omega t}$\\
  $\gamma$ is the drive strength, or $\frac{F_0}{mg}$ the ratio of the drive amplitude
  to the weight of the pendulum.\\
  \hl{Do I know what all the components of this equation mean?}\\

  \notecard{$\ddot{\phi} + 2 \beta \dot{\phi} + \omega_0^2 \sin{\phi} = \gamma \omega_0^2 \cos{\omega t}$}


  \subsection{Linear and nonlinear systems}
  A differential equation is \textbf{linear} if and only if it involves the dependent
  variable and its derivatives only linearly. A harmonic oscillator is linear,
  since its equation of motion depends only linearly on the oscillator's position,
  regardless of the form of the driving force. Conversely, a pendulum is nonlinear
  since it has a $\sin{\phi}$ dependence on the equation of motion.\\

  Linear systems are actually more the exception than the rule in terms of ubiquity.

   \notecard{Pendulum equation of motion: $mL^2\ddot{\phi} = -m g L \sin{\phi}$}

   \subsection{Chaotic and non-chaotic motion}
   Chaotic motion is highly sensitive to initial conditions, and requires nonlinearity,
   and a "somewhat complicated" equation of motion. It is also common in nonlinear systems.\\

   Plotting $\delta\phi$ for a chaotic system will show it exponentially increase.
   For a linear system, you will see it decay exponentially. This characterizes
   sensitivity to initial conditions.

   \notecard{A linear cannot exhibit chaos, but nonlinearity does not guarantee
   chaos.}
   \notecard{Nonlinear equations do not obey superposition.}
   \notecard{$\delta\phi$ characterizes sensitivity to initial conditions.}

   \subsection{Periodic and non-periodic motion}
   Straightforward. Be aware of the period doubling cascade.

   \subsection{Requirements for chaotic motion}
   \begin{itemize}
     \item Nonlinearity
     \item "Complicated equation of motion"
     \item Sensitivity to initial conditions
     \item At least 3 dimensions of state space -- \textbf{this is actually pretty wild}
   \end{itemize}


   \subsection{Lyapunov Exponent: Definition, use, and implications}
   The $\delta\phi$ between two identical DDPs with different I.C.s can be
   characterized as $|\delta\phi(t)| \propto Ke^{\lambda t}$. $\lambda$ is the
   Lyapunov exponent.\\

   If the long-term  motion is nonchaotic, this exponent is negative. If it is chaotic,
   then this is positive.\\

   \notecard{$|\delta\phi(t)| \propto Ke^{\lambda t}$. $\lambda$ is the
   Lyapunov exponent.}
   \notecard{$\lambda < 0$: nonchaotic, $\lambda > 0$: chaotic}

   \subsection{Feigenbaum Number}
   The interval between control parameter thresholds that kick the system
   into the next period state is given by the geometric progression (Feigenbaum Relation)
   $(\gamma_{n+1} - \gamma_n) \approx \frac{1}{\delta}(\gamma_n - \gamma_{n-1})$.

   \notecard{The inverse of the Feigenbaum number is the scaling factor between
   bifurcation points.}

   \subsection{Finite limit of a series}
   \hl{?}

   \subsection{Representations of motion}
   Plots 4 dayz



% \end{itemize}
%
% \subsection{Notecard}

\end{document}
